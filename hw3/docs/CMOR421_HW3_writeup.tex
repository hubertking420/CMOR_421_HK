\documentclass{article}
\usepackage{graphicx} % Required for inserting images
\usepackage{listings} % Required for insertion of code
\usepackage{amsmath} % Required for math features
\usepackage{hyperref} % Required for adding hyperlinks
\usepackage{booktabs} % Required for better table formats
\usepackage{geometry}\geometry{margin=1in}
\usepackage{placeins} % Required to use \FloatBarrier to ensure figures are shown before section ends
\usepackage{amssymb} % Required for math symbols
\usepackage{algorithm}
\usepackage{algpseudocode}

\title{CMOR 421 Homework 3: MPI}
\author{Hubert King}
\date{April 9th, 2024}
\begin{document}
\maketitle

\section{Simplified SUMMA Algorithm}
For our simplified SUMMA, we use the following approach. Algorithm 1 provides a more detailed outline.
\begin{itemize}
    \item Scatter partitions of $A$ and $B$ to the processes, stored as $A^{local}$ and $B^{local}$. 
    \item Set $C^{local} \gets 0$.   
    \item Set $blocksize \gets n \div \sqrt{s}$.
    \item Allocate buffers $A^{buffer}$ and $B^{buffer}$.
    \item Initialize row and column communicators.
    \item In the main computational loop, all processes perform $p$ iterations of the following: 
\begin{itemize}
    \item Broadcast $A^{local}$ across its row.
    \item Broadcast $B^{local}$ across its column.
    \item Receive row broadcast into $A^{buffer}$.
    \item Receive column broadcast into $B^{buffer}$.
    \item Accumulate $C^{local} \gets A^{buffer} \times B^{buffer}$.
\end{itemize} 
\item Gather blocks of $C$ from all processes onto root process.
\end{itemize}

\section{Cannon's Algorithm}
For Cannon's algorithm, we use the following approach, keeping the same $blocksize$ from SUMMA and procedure for the initial scatter of the partitions of $A$, $B$, and $C$.
\begin{itemize}
    \item For each row $i$ of $A$, perform a left-rotating shift of the partitions by $i$ position.
    \item For each row $j$ of $B$, perform a upper-rotating shift of the partitions by $j$ position.
    \item In the main computational loop, all processes perform $p$ iterations of the following:
    \begin{itemize}
        \item Accumulate $C^{local} \gets A^{local} \times B^{local}$
        \item Perform left-circular shift of A on all rows by 1 position.
        \item Perform upward-circular shift of B on all columns by 1 position.
    \end{itemize}
    \item Gather blocks of C from all processes onto the root process.
\end{itemize}




\begin{algorithm}
\caption{SUMMA Algorithm Pseudocode}
\begin{algorithmic}[1]
\State \textbf{Input:} $A$, $B$, $rank$, $size$
\State \textbf{Output:} $C$
\State $blocksize \gets n / \sqrt{p}$
\If{$rank = 0$}
    \For{$k = p-1$ \textbf{to} $0$}
        \For{$i = 1$ \textbf{to} $blocksize$}
            \For{$j = 1$ \textbf{to} $blocksize$}
                \State $A^{local}_{ij} \gets A_{ij}$
                \State $B^{local}_{ij} \gets B_{ij}$
                \State $C^{local}_{ij} \gets 0$
            \EndFor
        \EndFor
        \If{$k > 0$}
            \State Send $A^{local}, B^{local}, C^{local}$ to process $k$.
        \EndIf
    \EndFor
\EndIf
\If{$rank > 0$}
    \State Receive $A^{local}, B^{local}, C^{local}$ from $rank = 0$.
\EndIf
\For{$k = 0$ \textbf{to} $\sqrt{p}-1$}
    \For{$i = 1$ \textbf{to} $\sqrt{p}$}
        \State Process holding block $A(i,k)$ broadcasts to its row.
    \EndFor
    \For{$j = 1$ \textbf{to} $\sqrt{p}$}
        \State Process holding block $B(k, j)$ broadcasts to its column.
    \EndFor
    \State Receive block $A(i,k)$ into $A^{buffer}$.
    \State Receive block $B(k,j)$ into $B^{buffer}$.
    \For{$i = 1$ \textbf{to} $blocksize$}
        \For{$j = 1$ \textbf{to} $blocksize$}
            \State $C^{local}_{ij} \gets C^{local}_{ij} + A^{buffer}_{ij} \cdot B^{buffer}_{ij}$
        \EndFor
    \EndFor
\EndFor
\If{$rank > 0$}
    \State Send $C^{local}$ to $rank = 0$.
\EndIf
\If{$rank = 0$}
    \State $C \gets C^{local}$
    \For{$k = size$ \textbf{to} $0$}
        \State Receive $C^{local}$ from $rank = k$
        \State $C \gets C^{local}$
    \EndFor
\EndIf
\State \textbf{return} $C$
\end{algorithmic}
\end{algorithm}
\FloatBarrier

\section{Miscellaneous Details}
To generate a random matrices for testing, we utilize the random number generation engine provided by the C++ standard library. We perform a correctness check by comparing product $C$ from the serial and parallel algorithms element-wise, with a tolerance of 1e-9. Timing results are measured in milliseconds.

\section{Build and Run Instructions}
Access NOTS via a login node and load the necessary modules:
\begin{verbatim}
module load GCCcore/13.2.0
module load OpenMPI
\end{verbatim}
Verify that the module is loaded correctly and the correct version of GCC is being used:
\begin{verbatim}
gcc --version
mpic++ --version
\end{verbatim}
Next, compile the drivers with the following command:
\begin{verbatim}
mpic++ -o summa -Iinclude summa.cpp src/functions.cpp
mpic++ -o cannon -Iinclude cannon.cpp src/functions.cpp
\end{verbatim}
After successful compilation, the programs can be tested by running the following command in the login node, where dimension is replaced by the desired matrix size.
\begin{verbatim}
mpirun -n <processors> cannon <dimension>
mpirun -n <processors> summa <dimension>
\end{verbatim}
To run, we use the following script, named job.slurm, which requests resources and runs the program on NOTS:
\begin{verbatim}
#!/bin/bash 
#SBATCH --job-name=CMOR-421-521
#SBATCH --partition=scavenge
#SBATCH --reservation=CMOR421
#SBATCH --ntasks=<requested-processors> 
#SBATCH --mem-per-cpu=1G 
#SBATCH --time=00:30:00 
echo "My job ran on:" 
echo $SLURM_NODELIST 
srun -n <processors> summa <dimension>
srun -n <processors> cannon <dimension>
\end{verbatim}
Submit the job with the following command:
\begin{verbatim}
sbatch job.slurm
\end{verbatim}
After job completion, view the output with the following command:
\begin{verbatim}
cat slurm-<job-number>.out
\end{verbatim}
Sample output:
\begin{verbatim}
My job ran on:
bc8u27n1
Matrix size n = 1024
Serial elapsed time = 8132.66
Elapsed time = 2.37754
Serial product and Cannon's product are equal to machine precision.
Matrix size n = 1024
Serial elapsed time = 8123.66
Elapsed time = 2.37487
Serial product and SUMMA product are equal to machine precision.
\end{verbatim}
\end{document}
